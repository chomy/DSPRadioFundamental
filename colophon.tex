\section*{編集後記}
DSPラジオの原理をお送りします。これはコミケ等で頒布した、
雑音工房偽術部彙報に、
私の個人blog noisefactory (\texttt{http://www.k.nakao.name/blog})に
書いた記事(無線LANデバイスが技術適合証明を受けなければならない理由)を追加したものです。

十分にチェックしたつもりですが、計算やロジックに間違いがある可能性が
あります。もし間違いがありましたらそれは100\%私の責任です。
何かありましたら、下記のtwitterアカウント、もしくはgithubリポジトリ
(ihttps://github.com/chomy/DSPRadioFundamental)
のIssueでお知らせください。


さて今後ですが、
夏コミには、「NginxでLuaLuaする」というタイトルで、NginxのLua拡張を使って
何かWebサービスを作って見た的なものを出そうと考えています。
(まだ1文字も書いていませんが、もしかしたらこのイベントに間に合うかもしれません)
また先日、私事ですが学者を廃業してIT セキュリティの世界に入りましたので
セキュリティ絡みで何か書けないかなと画策しております。

最後に、この同人誌はDebian/GNU Linux、\TeX Live2015、
psutils、git、GNU Make、vimといった、オープンソースソフトウェアを使って作成されました。
また日本語のフォントはIPAexフォントをPDFに埋め込んでいます。
このような有益なソフトウェアを開発、維持、管理していただいているすべての皆様に感謝します。
また、このページまでたどり着いてくれた読者の方(おそらくあなただけです)に感謝します。
ありがとうございました。

\begin{flushright}
2016年6月 Keisuke Nakao (@jm6xxu) 
\end{flushright}
\clearpage
\subsection*{参考文献}
\begin{itemize}
  \item 中島将光
    「マイクロ波工学」 森北出版 ISBN4-627-71030-5
  \item 常川光一
    「線状アンテナから電波が出るしくみ」 CQ出版社 RFワールド No.11 pp.30-39, ISBN978-4-7898-4890-9

  \item 
    SI4825-A10データシート \texttt{http://www.silabs.com/Support Documents/TechnicalDocs/Si4825-A10.pdf}
	\item
	法令データ提供サービス \texttt{http://law.e-gov.go.jp/cgi-bin/idxsearch.cgi} 電波法、電波法施行規則
   \end{itemize}
\clearpage
\mbox{}
%\clearpage
%\mbox{}
\vspace{36em}\\
この作品はクリエイティブ・コモンズ・ライセンス 表示 - 継承 2.1 日本 の下に提供されています。このライセンスのコピーを見るためには、http://creativecommons.org/licenses/by-sa/2.1/jp/ をご覧ください。
