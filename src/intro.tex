\chapter*{はじめに}

アラフォーオヤジが若い頃、PCはまだ非常に高価で、とても子供の手に届くものではありませんでした。
(私が母親との賭けで勝って最初に買ってもらったPCは40万円以上しました。
勤続20年の小学校教諭の夏のボーナスが飛ぶといったら想像できるでしょうか。)

そんな時代、技術好きの少年たちはラジオに夢中でした。
エアチェックといって、ただラジオ放送を録音する事にはじまり、
遠くの放送局から放送されている放送を聴くために、アンテナや受信機を
いじり始め、最後は自作するところまで。
(九州の田舎者が文化放送を聴くためには、技術力が必要だったのです。)
ハンダ付けを習得したのはラジオを
作るのが目的だった方も多いのではないでしょうか。

アナログからデジタルの時代になって久しいですが、ラジオも例外ではなく、
アンテナで受信した信号を数値に変換し、信号処理用のコンピュータ(DSP)を
使って復調されています。
つまり、昔ながらのコイルとコンデンサで作られた共振回路ではなく、
数値計算を使って、放送局から送信された電波から音声信号が取り出されています。
そしてその数値計算するためのコンピュータや、プログラム、周辺機器の多くは
ICチップに実装され、たった数百円で購入できます。

今回は、AMラジオの変調から、DSPの中で行われいるIQ検波の理論的な部分をご紹介します。


