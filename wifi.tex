\chapter*{おまけ\\Wifiに技適が必要な理由}

\section*{はじめに}
Wifiインターフェースを搭載したRaspberry Pi 3 model Bが発売されました。
まだ技適を受けていないので個人輸入は控えたほうが良い(注:現在は取得済み)という書き込みを見て、Wifiデバイスが技適を受けなければならない法的根拠を調べてみました。 以下、引用は法令データ提供サービスからで、強調は引用者が行ったものです。

\section*{免許が不要な無線局}

まず無線局を開設する時、原則総務大臣から免許を受けなければなりません。これは電波法第4条に規定されています。
\begin{quotation}
    電波法 第四条  無線局を開設しようとする者は、総務大臣の免許を受けなければならない。ただし、\textbf{次の各号に掲げる無線局については、この限りでない。}

    (中略)

    三  空中線電力が一ワット以下である無線局のうち\textbf{総務省令で定めるものであつて}、次条の規定により指定された呼出符号又は呼出名称を自動的に送信し、又は受信する機能その他総務省令で定める機能を有することにより他の無線局にその運用を阻害するような混信その他の妨害を与えないように運用することができるもので、\textbf{かつ、適合表示無線設備のみを使用するもの}
\end{quotation}

電波法第4条は、無線局の開設には免許が必要である事を規定していますが、同時に免許が不要な無線局についても規定しています。ここに規定されている「総務省令で定めるもの」は、電波法施行規則に規定されています。
\begin{quotation}
    電波法施行規則 第六条 法第四条第一号に規定する発射する電波が著しく微弱な無線局を次のとおり定める。

     (中略)

    4  法第四条第三号の総務省令で定める無線局は、次に掲げるものとする。

     (中略)

    四  主としてデータ伝送のために無線通信を行うもの(電気通信回線設備に接続するものを含む。)であつて、次に掲げる周波数の電波を使用し、かつ、空中線電力が〇・五八ワット以下であるもの(以下「小電力データ通信システムの無線局」という。)

    (1) 二、四〇〇MHz以上二、四八三・五MHz以下の周波数

    (2) 二、四七一MHz以上二、四九七MHz以下の周波数

    (3) 五、一五〇MHzを超え五、三五〇MHz以下の周波数(屋内その他電波の遮蔽効果が屋内と同等の場所であつて、総務大臣が別に告示する場所において使用するものに限る。)

    (4) 五、四七〇MHzを超え五、七二五MHz以下の周波数(上空にあつては、航空機内で運用する場合に限る。)

     (以下略)
\end{quotation}

無線LANで使用されている周波数の2.4GHz帯は各チャンネルの中心周波数2412MHz〜2472MHzで、専有周波数帯幅は20MHz(片側10MHz)なので、(1)に該当します。また5GHz帯は、各チャンネルの中心周波数5180MHz〜5320MHzと5500MHz〜5700MHzで、専有周波数帯幅が20MHzなので、(3)と(4)に該当します。 よって、電波法第4条3項の規定により、適合表示無線設備すなわち、\textbf{技術適合証明を受けた機器のみを使用する場合に限り、無免許で使用することができます}。 逆に言えば、技適を受けていない無線LANデバイスを日本国内で使用すると、免許を受けずに無線局を開設したことになり、電波法 第百十条により、\textbf{1年以下の懲役または100万円以下の罰金に処せられます}。

\section*{無線LANを無線従事者でなくても操作できる理由}

そういえば、無線局とはなんでしょう。無線局は電波法第2条で定義されています。
\begin{quotation}
    電波法 第二条  この法律及びこの法律に基づく命令の規定の解釈に関しては、次の定義に従うものとする。

     (中略)

    四  「無線設備」とは、無線電信、無線電話その他電波を送り、又は受けるための電気的設備をいう。

    五  「無線局」とは、無線設備及び無線設備の操作を行う者の総体をいう。但し、受信のみを目的とするものを含まない。
\end{quotation}
電波法第二条第四号により、無線LANデバイスが無線設備であることは明らかです。 また第五号で、無線局は無線設備と無線設備の操作を行う者の集合であると規定されています。さらに電波法第三十九条には、無線操作を行う者は、原則無線従事者免許を受けた者でなければならないことが規定されています。
\begin{quotation}
    電波法 第三十九条  第四十条の定めるところにより無線設備の操作を行うことができる無線従事者以外の者は、無線局の無線設備の操作(簡易な操作であつて総務省令で定めるものを除く。)を行つてはならない。(一部略)

どうやら、「簡易な操作であって総務省令で定めるもの」であれば、無線従事者でなくても良いようです。この簡易な操作は、電波法施行規則にありました。

    電波法施行規則 第三十三条  法第三十九条第一項本文の総務省令で定める簡易な操作は、次のとおりとする。ただし、第三十四条の二各号に掲げる無線設備の操作を除く。

    一  法第四条第一号から第三号までに規定する免許を要しない無線局の無線設備の操作

    (略)
\end{quotation}

第一号で無線従事者でなくとも操作ができるのは電波法第四条第一号から第三号までと規定されています。無線LANデバイスは、電波法第四条第三号に該当しますので、無線LANデバイスは無線従事者でなくても操作することができます。

\section*{陸上無線従事者の操作範囲に通信操作が含まれない理由}
余談ですが、無線従事者の資格の操作範囲は、通信操作と技術操作に分かれています。通信操作とは、例えばマイクやキーボードや電鍵を使って電文を送信することです。
一方技術操作は、アンテナを立てたり、無線機を設置、操作したり、調整したりといった操作です。これには無線設備を設計といったことから、無線機の電源を入れる事も含まれます。

実は、陸上の無線従事者の資格の操作範囲に通信操作が含まれていません。
以前から疑問に思っていたのですが、今回いろいろ調べていて、偶然、
陸上無線の操作範囲に 通信操作を含まない法的根拠を見つけました。


\begin{quotation}
    電波法施行規則 第三十三条  法第三十九条第一項本文の総務省令で定める簡易な操作は、次のとおりとする。ただし、第三十四条の二各号に掲げる無線設備の操作を除く。

     (中略)

    四  次に掲げる無線局(特定無線局に該当するものを除く。)の無線設備の通信操作

    (1) \textbf{陸上に開設した無線局}(海岸局、航空局、船上通信局、無線航行局及び海岸地球局並びに次号(4)の航空地球局を除く。)

    (2) 携帯局

    (3) 航空機地球局(航空機の安全運航又は正常運航に関する通信を行わないものに限る。)

    (4) 携帯移動地球局
\end{quotation}

なんと陸上の無線従事者の操作範囲に含まれるすべての通信操作は簡易な操作であり、無線従事者でなくても行うことができるとは。

